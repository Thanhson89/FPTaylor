\documentclass{article}
\usepackage{amsmath,amssymb,fouriernc,parskip,amsthm}
\usepackage{hyperref}
\hypersetup{
  citecolor=red,
  colorlinks=true
}

\begin{document}
\theoremstyle{definition}
\newtheorem{thm}{Theorem}[section]
\newtheorem{lem}[thm]{Lemma}

\section{Normalized Fractions/Exponents}

\begin{lem}
\label{pow}
Let
\begin{equation*}
S = \left \{ \; i \in \mathbb{Z}^{\geq 0} \; | \; r^{p - 1} \leq f \cdot r^i 
\; \right \}
\end{equation*}
Then
\begin{equation*}
\forall r, f, p \in \mathbb{N} \; . f > 0 \; \wedge \; p > 0 \; 
\Longrightarrow \; \exists i^* \in S \; . \forall i \in S \; . \; i^* \leq i
\end{equation*}
\begin{proof} Assume the antecedent. Since $f > 0$, $f \geq 1$. Moreover, 
$p > 0$, so $p - 1 \geq 0$ and $r^{p - 1} \leq f \cdot r^{p - 1}$. It follows
that $p - 1 \in S$. From the Well-Ordering Principle, $S$ has a smallest
element $i^*$. 
\end{proof}
\end{lem}

\begin{lem}
\label{posfrac}
\begin{equation*}
\forall x \in \mathbb{R}, f \in \mathbb{N}, e \in \mathbb{Z} \; . \; 
x \neq 0 \; \wedge \; frep(f,e,x) \; \Longrightarrow \; f > 0
\end{equation*}
\begin{proof} Assume the antecedent. Expanding the defn of $frep$, we know
\begin{equation*}
|x| = f \cdot r^{(e - p + 1)}
\end{equation*}
Since $x \neq 0$, $|x| > 0$. $r^{(e - p + 1)}$ is always positive, so $f > 0$.
\end{proof}
\end{lem}

\begin{thm}
\label{normfrac}
\begin{equation*}
\forall x \in \mathbb{R} \; . \; x \neq 0 \; \wedge \; is\_float(x) \; 
\Longrightarrow \; \exists f' \in \mathbb{N} \; . \; is\_norm\_frac(f', x)
\end{equation*}
\begin{proof} Assume $is\_float(x)$ and $x \neq 0$. From the defn of
$is\_float$, we know there exists $f \in \mathbb{N}$ and $e \in \mathbb{Z}$
such that $frep(f, e, x)$. From ~\ref{posfrac}, we know $f > 0$.

Also, from the defn of $frep$, we know
\begin{equation*}
f < r^p \; \wedge \; |x| = f \cdot r^{(e - p + 1)}
\end{equation*}
From ~\ref{pow}, we know there is a smallest integer $i^*$ such that
$r^{p - 1} \leq f \cdot r^{i^*}$. We also know that $f \cdot r^{i^*} < r^p$
(if it wasn't, we could cancel r on both sides and obtain a smaller
$i$, contradicting the definition of $i^*$). Take $f' = f \cdot r^{i^*}$.
It immediately follows that $f' \geq r^{p - 1}$. If we also take
$e' = e - i^*$, then
\begin{align*}
f' \cdot r^{e' - p + 1} &= f \cdot r^{i^*} \cdot r^{e - i^* - p + 1}\\
&= f \cdot r^{e - p + 1}\\
&= |x|
\end{align*}
and so $is\_norm\_frac(f', x)$ is true.
\end{proof}
\end{thm}

\subsection{Comments}

The proof for existence of normalized exponents is similar. Do we need 
uniqueness of normalized fractions/exponents? I think for now, probably not.

\section{largest/smallest}

\begin{thm}
\label{largestexists}
\begin{align*}
&\forall x \in \mathbb{R}, P:\mathbb{R} \to Bool \; . \\
&\qquad \big [ \; \exists y \in \mathbb{R} \; . \;
is\_float(y) \; \wedge \; P(y) \; \big ] \; \wedge \\
& \qquad \Big [ \; \exists b \in \mathbb{R} \; . \; 
b \neq 0 \; \wedge \; \big [ \; \forall y \in \mathbb{R} \; . 
\; is\_float(y) \; \wedge \; P(y) \; 
\Longrightarrow \; y \leq b \; \big ] \; \Big ] \; \Longrightarrow \\
& \qquad \exists y \in \mathbb{R} \; . \; is\_largest(P, y)
\end{align*}
\begin{proof} Assume the antecedent. 


\end{proof}
\end{thm}


\end{document}
